\documentclass[11pt]{amsart}

\usepackage{a4wide}
\usepackage{paralist}
\usepackage{url}
\usepackage{bbm}
\usepackage{nopageno}

\newcommand{\cA}{\mathcal{A}}
\newcommand{\cS}{\mathcal{S}}
\DeclareMathOperator{\conv}{conv}
\DeclareMathOperator{\New}{New}
\DeclareMathOperator{\area}{area}
\newcommand{\RR}{\mathbbm{R}}
\newcommand{\CC}{\mathbbm{C}}

\begin{document}
\begin{center}
\textbf{\sffamily
   Discrete and Algorithmic Geometry }

\medskip
   Julian Pfeifle,
   UPC, 2016
\end{center}

\bigskip

\begin{center}
  \textbf{\sffamily Sheet 2}

\bigskip
 due on Monday, November 28, 2016

\end{center}

\bigskip
\bigskip
\bigskip

\section*{Writing}

\begin{enumerate}
\item Matou\v sek, \emph{Lectures on Discrete Geometry}, Exercises 5.1.2, 5.1.3

\item Let $Q=C_4(7)^\Delta$, the polar dual of the 4-dimensional cyclic polytope on $7$~vertices.
  \begin{enumerate}
  \item Calculate $f(Q)=\big(f_0,f_1,f_2,f_3)$, the $f$-vector of~$Q$.
  \item How many combinatorially distinct facets does $Q$ have? Draw one example of each combinatorial type. Now select a vertex $q\in Q$ and draw all facets incident to~$q$.
  \end{enumerate}
\item Let $P$ be the \emph{24-cell}, $P=\conv\{ \pm e_i \pm e_j : 1\le i,j\le 4,\, i\ne j\}$ where $(e_1,\dots,e_4)$ is the standard basis of $\RR^4$. Describe the face lattice of~$P$, and prove that $P$~is \emph{self-polar-dual} (it has the same combinatorial type as its polar dual).
\end{enumerate}

\end{document}
 
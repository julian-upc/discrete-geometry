\chapter{Kalai's simple way of telling a simple polytope from its graph}

\scribe{Alex Alvarez}

\begin{conj}[Micha Perles, $\sim$1970]
Any $d$-polytope is determined by its graph, i.e., $sk^1(P)$ determines $\mathcal{F}(P)$.
\end{conj}

The conjecture is false, e.g. for simplicial polytopes. On the other hand, $K_n$ is dimensionally ambiguous, e.g. $sk^1(C_d(n)) = sk^1(C_e(n))$ for $d , e < n$. Besides there are neighborly graphs that are not combinatorially isomorphic to cyclic ones. However, the result is true for simple polytopes:

\begin{theorem}[Roswitha Blind, Peter Mani, 1987]
Any simple $d$-polytope is  determined by its graph.
\end{theorem}

Their proof is not constructive, but in 1988 Kalai found an algorithmic approach. Let us see the details of that approach.

\begin{definition}[Acyclic orientation]
An acyclic orientation of a graph $G$ is an orientation of $E(G)$ without directed cycles.
\end{definition}

\begin{definition}[Abstract Objective Function, AOF]
An AOF of $G = sk^1(P)$ is an acyclic orientation that has a unique sink on each face of $P$.
\end{definition}

\begin{definition}[Unique Sink Orientation]
A Unique Sink Orientation is an orientation such that every face has a unique sink. The orientation is not necessarily acyclic.
\end{definition}

Notice that AOFs exist because of the existence of generic linear objective functions. Consider the set of all AOFs and define the following function:

\begin{equation}
f(O) = h_0(O) + 2h_1(O) + \ldots + 2^kh_k(O) + \ldots + 2^dh_d(O)
\end{equation}
where $O$ is an AOF and $h_k(O)\in\mathbb{N}$ counts the vertices of in-degree $k$ in $O$.

\begin{figure}[!h]
\centering 
\centerline{%%Created by jPicEdt 1.4.1_03: mixed JPIC-XML/LaTeX format
%%Wed Dec 04 02:22:05 CET 2013
%%Begin JPIC-XML
%<?xml version="1.0" standalone="yes"?>
%<jpic x-min="0" x-max="40" y-min="0" y-max="40" auto-bounding="true">
%<parallelogram p3= "(40,0)"
%	 p2= "(40,40)"
%	 p1= "(0,40)"
%	 fill-style= "none"
%	 />
%<parallelogram p3= "(30,10)"
%	 p2= "(30,30)"
%	 p1= "(10,30)"
%	 fill-style= "none"
%	 />
%<multicurve arrow-bracket-length-scale= "0.25"
%	 arrow-global-scale-width= "1.5"
%	 fill-style= "none"
%	 arrow-rbracket-length-scale= "0.1"
%	 points= "(0,40);(0,40);(5,35);(5,35)"
%	 right-arrow= "head"
%	 arrow-head-width-minimum= "1.6"
%	 />
%<multicurve arrow-bracket-length-scale= "0.25"
%	 arrow-global-scale-width= "1.5"
%	 fill-style= "none"
%	 arrow-rbracket-length-scale= "0.1"
%	 points= "(0,0);(0,0);(0,20);(0,20)"
%	 right-arrow= "head"
%	 arrow-head-width-minimum= "1.6"
%	 />
%<multicurve arrow-bracket-length-scale= "0.25"
%	 arrow-global-scale-width= "1.5"
%	 fill-style= "none"
%	 arrow-rbracket-length-scale= "0.1"
%	 points= "(10,10);(10,10);(10,20);(10,20)"
%	 right-arrow= "head"
%	 arrow-head-width-minimum= "1.6"
%	 />
%<multicurve arrow-bracket-length-scale= "0.25"
%	 arrow-global-scale-width= "1.5"
%	 fill-style= "none"
%	 arrow-rbracket-length-scale= "0.1"
%	 points= "(0,0);(0,0);(5,5);(5,5)"
%	 right-arrow= "head"
%	 arrow-head-width-minimum= "1.6"
%	 />
%<multicurve arrow-bracket-length-scale= "0.25"
%	 arrow-global-scale-width= "1.5"
%	 fill-style= "none"
%	 arrow-rbracket-length-scale= "0.1"
%	 points= "(0,0);(0,0);(20,0);(20,0)"
%	 right-arrow= "head"
%	 arrow-head-width-minimum= "1.6"
%	 />
%<multicurve arrow-bracket-length-scale= "0.25"
%	 arrow-global-scale-width= "1.5"
%	 fill-style= "none"
%	 arrow-rbracket-length-scale= "0.1"
%	 points= "(40,0);(40,0);(35,5);(35,5)"
%	 right-arrow= "head"
%	 arrow-head-width-minimum= "1.6"
%	 />
%<multicurve arrow-bracket-length-scale= "0.25"
%	 arrow-global-scale-width= "1.5"
%	 fill-style= "none"
%	 arrow-rbracket-length-scale= "0.1"
%	 points= "(40,0);(40,0);(40,20);(40,20)"
%	 right-arrow= "head"
%	 arrow-head-width-minimum= "1.6"
%	 />
%<multicurve arrow-bracket-length-scale= "0.25"
%	 arrow-global-scale-width= "1.5"
%	 fill-style= "none"
%	 arrow-rbracket-length-scale= "0.1"
%	 points= "(30,10);(30,10);(30,20);(30,20)"
%	 right-arrow= "head"
%	 arrow-head-width-minimum= "1.6"
%	 />
%<multicurve arrow-bracket-length-scale= "0.25"
%	 arrow-global-scale-width= "1.5"
%	 fill-style= "none"
%	 arrow-rbracket-length-scale= "0.1"
%	 points= "(10,10);(10,10);(20,10);(20,10)"
%	 right-arrow= "head"
%	 arrow-head-width-minimum= "1.6"
%	 />
%<multicurve arrow-bracket-length-scale= "0.25"
%	 arrow-global-scale-width= "1.5"
%	 fill-style= "none"
%	 arrow-rbracket-length-scale= "0.1"
%	 points= "(10,30);(10,30);(20,30);(20,30)"
%	 right-arrow= "head"
%	 arrow-head-width-minimum= "1.6"
%	 />
%<multicurve arrow-bracket-length-scale= "0.25"
%	 arrow-global-scale-width= "1.5"
%	 fill-style= "none"
%	 arrow-rbracket-length-scale= "0.1"
%	 points= "(0,40);(0,40);(20,40);(20,40)"
%	 right-arrow= "head"
%	 arrow-head-width-minimum= "1.6"
%	 />
%<multicurve fill-style= "none"
%	 points= "(30,30);(30,30);(35,35);(35,35)"
%	 />
%<multicurve fill-style= "none"
%	 points= "(0,0);(0,0);(10,10);(10,10)"
%	 />
%<multicurve fill-style= "none"
%	 points= "(40,0);(40,0);(30,10);(30,10)"
%	 />
%<multicurve fill-style= "none"
%	 points= "(30,30);(30,30);(40,40);(40,40)"
%	 />
%<multicurve fill-style= "none"
%	 points= "(0,40);(0,40);(10,30);(10,30)"
%	 />
%<multicurve arrow-bracket-length-scale= "0.25"
%	 arrow-global-scale-width= "1.5"
%	 fill-style= "none"
%	 arrow-rbracket-length-scale= "0.1"
%	 points= "(30,30);(30,30);(35,35);(35,35)"
%	 right-arrow= "head"
%	 arrow-head-width-minimum= "1.6"
%	 />
%</jpic>
%%End JPIC-XML
%LaTeX-picture environment using emulated lines and arcs
%You can rescale the whole picture (to 80% for instance) by using the command \def\JPicScale{0.8}
\ifx\JPicScale\undefined\def\JPicScale{1}\fi
\unitlength \JPicScale mm
\begin{picture}(40,40)(0,0)
\linethickness{0.3mm}
\put(0,40){\line(1,0){40}}
\put(0,0){\line(0,1){40}}
\put(40,0){\line(0,1){40}}
\put(0,0){\line(1,0){40}}
\linethickness{0.3mm}
\put(10,30){\line(1,0){20}}
\put(10,10){\line(0,1){20}}
\put(30,10){\line(0,1){20}}
\put(10,10){\line(1,0){20}}
\linethickness{0.3mm}
\multiput(0,40)(0.12,-0.12){42}{\line(1,0){0.12}}
\put(5,35){\vector(1,-1){0.12}}
\linethickness{0.3mm}
\put(0,0){\line(0,1){20}}
\put(0,20){\vector(0,1){0.12}}
\linethickness{0.3mm}
\put(10,10){\line(0,1){10}}
\put(10,20){\vector(0,1){0.12}}
\linethickness{0.3mm}
\multiput(0,0)(0.12,0.12){42}{\line(1,0){0.12}}
\put(5,5){\vector(1,1){0.12}}
\linethickness{0.3mm}
\put(0,0){\line(1,0){20}}
\put(20,0){\vector(1,0){0.12}}
\linethickness{0.3mm}
\multiput(35,5)(0.12,-0.12){42}{\line(1,0){0.12}}
\put(35,5){\vector(-1,1){0.12}}
\linethickness{0.3mm}
\put(40,0){\line(0,1){20}}
\put(40,20){\vector(0,1){0.12}}
\linethickness{0.3mm}
\put(30,10){\line(0,1){10}}
\put(30,20){\vector(0,1){0.12}}
\linethickness{0.3mm}
\put(10,10){\line(1,0){10}}
\put(20,10){\vector(1,0){0.12}}
\linethickness{0.3mm}
\put(10,30){\line(1,0){10}}
\put(20,30){\vector(1,0){0.12}}
\linethickness{0.3mm}
\put(0,40){\line(1,0){20}}
\put(20,40){\vector(1,0){0.12}}
\linethickness{0.3mm}
\multiput(30,30)(0.12,0.12){42}{\line(1,0){0.12}}
\linethickness{0.3mm}
\multiput(0,0)(0.12,0.12){83}{\line(1,0){0.12}}
\linethickness{0.3mm}
\multiput(30,10)(0.12,-0.12){83}{\line(1,0){0.12}}
\linethickness{0.3mm}
\multiput(30,30)(0.12,0.12){83}{\line(1,0){0.12}}
\linethickness{0.3mm}
\multiput(0,40)(0.12,-0.12){83}{\line(1,0){0.12}}
\linethickness{0.3mm}
\multiput(30,30)(0.12,0.12){42}{\line(1,0){0.12}}
\put(35,35){\vector(1,1){0.12}}
\end{picture}
}

\caption{An example of AOF. For this orientation, $h_0 = 1$, $h_1 = 3$, $h_2 = 3$ and $h_3 = 1$. Therefore, $f(O) = 1 + 6 + 12 + 8 = 27 = 3^3$.}
\end{figure}

Let $f$ denote the number of non-empty faces of $P$. Given an acyclic orientation $O$, then

\begin{enumerate}
\item $f(O) \geq f$, because as $O$ is acyclic each face has at least one sink.
\item $f(O) = f$ if $O$ is an AOF.
\end{enumerate}

To determine all AOFs of $P$ just from its graph $G = sk^1(P)$ do the following steps:
\begin{enumerate}
\item Enumerate all acyclic orientations $O$ of $G$.
\item Calculate $f(O)$ for each such orientation $O$.
\item Keep the orientations $O$ with minimal $f(O)$.
\end{enumerate}

The only remaining step to complete Kalai's method is showing a way to identify the faces of $P$ from the knowledge of all the AOFs. Notice that the subgraphs of $G$ that are graphs of faces are connected, $k$-regular and induced. With that, the following proposition is enough to characterize the faces:

\begin{proposition}
An induced connected $k$-regular subgraph $H$ is the graph of a face of $P$ $\Leftrightarrow$ $H$ is an initial set for some AOF $O$.
\end{proposition}
\begin{proof}

$\Rightarrow$: Perturb a face-defining inequality to obtain a linear function with respect to which the vertices of $F$ lie below all other vertices.

$\Leftarrow$: Let $H$ be a connected $k$-regular subgraph of $sk^1(P)$ and let $O$ be an AOF such that $H$ is an initial set of $O$. As $H$ has a unique sink $v$ and is $k$-regular, the sink $v$ has $k$ incoming edges and as the polytope is simple, those $k$ edges determine a $k$-face $F$ and $v$ is a sink in that face.

As $v$ is the unique sink of $O$ in $F$, all vertices of $F$ are $\leq v$ with respect to $O$. But the initial set of $v$ in $G$ is $H$, so vert$(F) \subseteq $ vert$(H)$. As both $H$ and $F$ are connected and $k$-regular, we have that vert$(F) = $ vert$(H)$.
\end{proof}

Notice that this algorithm is exponential, but in 1995 Friedman gave a polynomial algorithm to reconstruct the 2-faces, which had been proven before to be enough to reconstruct the whole polytope.
 

% Local Variables: 
% mode: latex
% TeX-master: "dag-upc"
% End: 

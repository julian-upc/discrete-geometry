\chapter*{Scribes 2013}


\section{Cecilia Girón Albert}

I've got my Degree in Mathematics at the Universidad Autónoma de Madrid, which I completed my fifth and sixth semesters 
at the University of Jyväskyla (Finland) as an Erasmus student. My degree has been mainly focused on subjects like 
analysis, statistics and numerical methods, although I am more interested in algebra and graph theory. 

I decided to study the Master in Advanced Mathematics and Mathematical Engineering to keep developing my mathematics skills 
in some theoretical subjects that can be applied in real life problems. Therefore, I believe that this course is a great 
opportunity to learn more about algorithmic and computing science and even more importantly, it may help me to find out 
what field I would like to focus on in the future.

\section{Anna Somoza}

I have recently finished a degree in Mathematics at the Universitat Politècnica de Catalunya. During this degree I developed a great interest in Algebra fields. In particular, I took the optional subjects \emph{Algebraic Geometry}, \emph{Algebraic Topology} and \emph{Galois Theory} and I wrote my Final Degree Thesis on a topic of Number Theory.

Now I'm taking the Master in Advanced Mathematics and Mathematical Engineering to develop my knowledge in these and other fields, and I my aim is to start a PhD in Number Theory next year. I enroled this subject because I have allways liked both computer science and geometry, and it seemed to be interesting. Therefore, I would be interested in the topic related to algebraic geometry.

\section{Daniel Torres}

My name is Daniel Torres and I am graduate in mathematics in UPC. Along the degree I have developed much interest in fields of topology, algebra and geometry, and some loathing to study (not to programming) numerical methods and modelling. I decided study this master, and particularly this subject, for expand my knowledge about my interests.

More concretely, I am doing this course with the hope it shows me about geometry.




% Local Variables: 
% mode: latex
% TeX-master: "dag-upc"
% End: 

\scribe{Ane Santos}



\begin{definition}
Informally, an \emph{orbifold} is the quotient of a manifold (here, the Euclidean plane) by the action of a group.
\end{definition}

\begin{displaymath}
	\begin{array}{lcc}
		torus & \longrightarrow & \circ \\
		holes & \longrightarrow & * \\
		non-orientability & \longrightarrow & \times \\
		boundary singularity & \longrightarrow & *n \\
		core point of order n & \longrightarrow & n \
	\end{array}
\end{displaymath}

\begin{theorem}{Magic theorem for the sphere}
The total cost of the signature of any spherical group is $ \$ 2-frac{2}{g} $ where $g=total number of symmetries$
\end{theorem}

The Magic theorem in the plane is a special case because the number of symmetries in a plane is infinite, so the cost is always 2.

There are 14 spherical symmetry groups: $m,n\geq1$

\begin{displaymath}
	\begin{array}{lcccc}
		*532 & *432 & *332 & *22n & *mn \\
		 & & 3*2 & 2*n & n* \\
		 & & & & n\times \\
		 532 & 432 & 332 & 22n & mn 
	\end{array}
\end{displaymath}

If $n\rightarrow\infty$ and $m\rightarrow\infty$ in $*22n$, $*mn$, $2*n$, $n*$, $n\times$, $22n$ and $mn$ we get the 7 possible groups of friezes (cenefas).

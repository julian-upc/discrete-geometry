\documentclass[draft,11pt]{amsbook}

\makeatletter
\def\@thm#1#2#3{%
  \ifhmode\unskip\unskip\par\fi
  \normalfont
  \trivlist
  \let\thmheadnl\relax
  \let\thm@swap\@gobble
  \let\thm@indent\indent % indent
  \thm@headfont{\bfseries}% heading font boldface // changed
  \thm@notefont{\fontseries\mddefault\upshape}%
  \thm@headpunct{.}% add period after heading
  \thm@headsep 5\p@ plus\p@ minus\p@\relax
  \thm@space@setup
  #1% style overrides
  \@topsep \thm@preskip               % used by thm head
  \@topsepadd \thm@postskip           % used by \@endparenv
  \def\@tempa{#2}\ifx\@empty\@tempa
    \def\@tempa{\@oparg{\@begintheorem{#3}{}}[]}%
  \else
    \refstepcounter{#2}%
    \def\@tempa{\@oparg{\@begintheorem{#3}{\csname the#2\endcsname}}[]}%
  \fi
  \@tempa
}
\makeatother

\renewcommand{\chaptername}{Lecture}

\usepackage[T1]{fontenc}
\usepackage[latin1]{inputenc}
\usepackage{times}
\usepackage{microtype}
\usepackage{amssymb}
\usepackage{a4wide}
\usepackage{graphicx}
%\usepackage{paralist}
%\usepackage{bbm}
\usepackage{verbatim}
\usepackage{url}

\newcommand{\ba}{{\boldsymbol{a}}}
\newcommand{\balpha}{{\boldsymbol{\alpha}}}
\newcommand{\bb}{{\boldsymbol{b}}}
\newcommand{\bc}{{\boldsymbol{c}}}
\newcommand{\be}{{\boldsymbol{e}}}
\newcommand{\bff}{{\boldsymbol{f}}}
\newcommand{\bnu}{{\boldsymbol{\nu}}}
\newcommand{\bm}{{\boldsymbol{m}}}
\newcommand{\bo}{{\boldsymbol{0}}}
\newcommand{\bp}{{\boldsymbol{p}}}
\newcommand{\bq}{{\boldsymbol{q}}}
\newcommand{\br}{{\boldsymbol{r}}}
\newcommand{\bsigma}{{\boldsymbol{\sigma}}}
\newcommand{\bt}{{\boldsymbol{t}}}
\newcommand{\bv}{{\boldsymbol{v}}}
\newcommand{\bw}{{\boldsymbol{w}}}
\newcommand{\bx}{{\boldsymbol{x}}}
\newcommand{\by}{{\boldsymbol{y}}}
\newcommand{\bz}{{\boldsymbol{z}}}
\newcommand{\bone}{{\boldsymbol{1}}}

\newcommand{\RR}{\mathbb{R}}
\newcommand{\Rgeo}{{\mathbb{R}_{\ge0}}}
\newcommand{\Zgeo}{{\mathbb{Z}_{\ge0}}}
\newcommand{\NN}{\mathbb{N}}
\newcommand{\ZZ}{\mathbb{Z}}
\newcommand{\QQ}{\mathbb{Q}}
\newcommand{\CC}{\mathbb{C}}

\newcommand{\cA}{{\mathcal{A}}}
\newcommand{\cC}{{\mathcal{C}}}
\newcommand{\cD}{{\mathcal{D}}}
\newcommand{\cH}{{\mathcal{H}}}
\newcommand{\cL}{{\mathcal{L}}}
\newcommand{\cO}{{\mathcal{O}}}
\newcommand{\cP}{{\mathcal{P}}}

\newcommand{\scp}[2]{\langle #1,#2\rangle}
\newcommand{\fl}[1]{\left\lfloor #1\right\rfloor}
\newcommand{\ce}[1]{\left\lceil #1\right\rceil}
\newcommand{\rcone}[1]{{{}_\Rgeo\!\left\langle #1\right\rangle}}
\newcommand{\zcone}[1]{{{}_\Zgeo\!\left\langle #1\right\rangle}}

\DeclareMathOperator{\interior}{int}
\DeclareMathOperator{\relint}{relint}
\DeclareMathOperator{\conv}{conv}
\DeclareMathOperator{\cone}{cone}
\DeclareMathOperator{\aff}{aff}
\DeclareMathOperator{\vol}{vol}
\DeclareMathOperator{\dist}{dist}
\DeclareMathOperator{\vertices}{vert}
\DeclareMathOperator{\rang}{rang}
\DeclareMathOperator{\sign}{sign}
\DeclareMathOperator{\pyr}{pyr}
\DeclareMathOperator{\bipyr}{bipyr}
\DeclareMathOperator{\Sl}{Sl}

\DeclareMathOperator{\sgn}{sgn} %signum
\DeclareMathOperator{\ggT}{ggT}

\newcommand{\ojo}[1]{\textsf{\bfseries\boldmath #1}}
\newcommand{\scribe}[1]{\begin{center}\emph{Scribe: #1}\end{center}\bigskip}

\graphicspath{{graphics/}}

\numberwithin{equation}{section}

\newtheorem{theorem}{\textbf{Theorem}}[section]
\newtheorem{lemma}[theorem]{Lemma}
\newtheorem{proposition}[theorem]{Proposition}
\newtheorem{corollary}[theorem]{Corollary}
\newtheorem{conj}[theorem]{Conjecture}
\newtheorem{obs}[theorem]{Observation}
\newtheorem{exercise}[theorem]{Exercise}
\newtheorem{example}[theorem]{Example}
\newtheorem{remark}[theorem]{Remark}

\newtheorem{defn}{\textbf{Definition}}[section]

\includeonly{lecture2}

\begin{document}

\thispagestyle{empty}

\
\vfill
\begin{center}
        \Huge \sffamily\bfseries 
        Discrete and Algorithmic Geometry 2011
        \medskip
        (Part 2)

\vspace{2cm}
\LARGE
Julian Pfeifle

\vspace{3cm}

\normalfont\LARGE\sffamily
Version of \today

\vspace{5cm}\
\end{center}

This is the preliminary version of the lecture notes for the second
part of \emph{Discrete and Algorithmic Geometry} (Universitat
Polit�cnica de Catalunya), held in the fall semester of 2011 by Vera
Sacristan and Julian Pfeifle.

\medskip
These notes are fruit of the collaborative effort of all participating
students, who have taken turns in assembling this text. The name of
each scribe figures in each corresponding section.

\medskip
The main literature for this course consists of
\cite{Conway-Sloane-3rd}, \cite{Conway-Strauss08}
and~\cite{Senechal95}. 

\medskip Suggestions for improvements will always be gladly received
by \texttt{julian.pfeifle@upc.edu}.

\vfill\


% Local Variables: 
% mode: latex
% TeX-master: "dag-2011"
% End: 

\tableofcontents

\chapter{Convex Polytopes}

\scribe{Cecilia Gir\'on}
 
A convex polytope can be defined in two different ways:
\begin{enumerate}
\item[-] \textit{$V$-polytope} (discrete geometry): is the convex hull of the finite non-empty point set in $\mathbf{R}^d$.
\item[-]\textit{$H$-polytope} (linear/integer optimization): An $H$-polyhedron is an intersection of a finite number of linear half spaces in some $\mathbf{R}^d$, if non-empty. And an $H$-polytope is a bounded $H$-polyhedron. 
\end{enumerate}
\bigskip
\section{Faces}
One of the properties studied about polytopes is their faces. A \textbf{face} $F$ of a polytope $P$ is a set of the form:
\[F = \{x\in \mathbb{R}^d: <a,x> = n\}\cap P\]
where $a\in (\mathbb{R}^d)^*$ (dual space), $b\in\mathbb{R}$ and $P\subseteq\{x\in\mathbb{R}^d: <a,x> \leq b \}\longleftrightarrow$ The inequality $<a,x> \leq b$ is valid for $P$. Notice that $P$ is actually a face of itself. 

We can also study the dimension $\dim F$ of a face. Let $P$ be a $d$ dimensional polytope, then if a face $F$ of $P$ has dimension:
\begin{itemize}
\item[i)] $d -1$, it is called a \textbf{facet}.
\item[ii)] $d -2$, it is called a \textbf{ridge}.
\item[iii)] $1$, it is called an \textbf{edge}.
\item[iv)] $0$, it is called a \textbf{vertex}.
\item[v)] $ -1$ then $F=\emptyset$.
\end{itemize}

The partially ordered set of all faces  $\mathcal{F}(P)$ of a convex polytope $P$ forms a Eulerian lattice called \textbf{face lattice}. The face lattice can be used, for instance, to count the number of faces of same dimension:
\[f_i = \# \{ \mbox{ faces } F \mbox{ of } P \mbox{ with } \dim F=i\} \label{eq1}\]

\bigskip

\textsc{Example}. Let $P$ be a cube in two dimensions, so a square. Notice that $\dim P = 2$, for every edge $ij\in P$ $\dim(ij) = 1$ and for every vertex $i$ $\dim i= 0$, for $i,j= 1, 2 , 3,4$ and $i\neq j$. With this information we can construct the face lattice and study some properties of $P$.

\begin{figure}[h!]
\begin{minipage}[t]{0.4\textwidth}
   \vspace{30pt}
   \hspace{20pt}

\begin{picture}(100,100)
\put(115,0){4}
\put(0,1){3}
\put(0,100){1}
\put(115,100){2}
\put(8,105){\textbullet}
\put(108,105){\textbullet}
\put(8,8){\textbullet}
\put(108,8){\textbullet}
\multiput(10,10)(100,0){2}{\line(0,100){100}}
\multiput(10,10)(0,100){2}{\line(100,0){100}}
\end{picture}

\end{minipage}
  \hfill
\begin{minipage}[t]{0.6\textwidth}
      \vspace{0pt}
      \hspace{-100pt}
\[
\xymatrix{
& & P\ar[dr]\ar[drr]\ar[dl]\ar[dll] & & & f_2 = 1 \\
13\ar[d]\ar[drrrr]  & 12\ar[d]\ar[dl]&  & 24\ar[d]\ar[dll] & 34\ar[d]\ar[dl] & f_ 1 = 4\\
1\ar[drr]  & 2\ar[dr]&  & 3\ar[dl] & 4\ar[dll] & f_0=4\\
 & & \emptyset & & & F_{-1}=1
 }
\]
\end{minipage}
\end{figure} 


\begin{flushright}
$\clubsuit$
\end{flushright}

\bigskip
\textsc{Example}.Let's study now the dimension of the faces of a hypercube of dimension d $\square ^d = \{ x\in\mathbb{R}^d: -1\leq x_i\leq 1, i=1,\cdots, d\}$: $f_{-1}(\square ^d) = 1, f_{0}(\square ^d) = 2^d,f_{d-1}(\square ^d) = 2d $ and $f_{d}(\square ^d) = 1$. Notice that the radius $r$ from the center of the cube to one of its vertices is $r=\sqrt{d}-1$, thus the exterior circus of the polytope has radio $r$ and the interior radio 1. 
\bigskip



\begin{tabular}{| c | c | c | c | c | c | c | c | c |}
  \hline                        
  d & 2 & 3 & 4 & 5 & $\cdots$ & 100 & $\cdots$ & $10^{100}$ \\
  \hline 
 $r=\sqrt{d}-1$ & $\sqrt{2}-1$ & $\sqrt{3}-1$ & $1$ & $\sqrt{5}-1$ & $\cdots$ & 9  & $\cdots$ & $10^{50}-1$ \\
  \hline  
  $f_0$ & $4$ & $8$ & $16$  & $32$ &  $\cdots$ & $2^{100}$ &  $\cdots$ & $2^{10^{100}}$ \\
  \hline
  $f_{1}$ & $4$ & $6$ & $8$  & $10$ &  $\cdots$ & $200$ &  $\cdots$ & $2\cdot 10^{100}$ \\
  \hline  
\end{tabular}

\begin{flushright}
$\clubsuit$
\end{flushright}

\bigskip

An other property that can be studied about the faces of convex polytopes is whether they are a simplex or not. Let $P$ be a polytope such that $\dim P = d$ and $\mathcal{F}(p) = k+1$. It is said to be \textbf{simplicial} if it is $k$-simplex, i.e. if each of its faces is a simplex; and it is called \textbf{simple} if each of its vertices is contained in exactly $d$ faces where $\dim P =d$.

\bigskip
\subsection{Exercises done during the lecture 8/11/2013. Each one includes one}

\begin{description}
\item [Alex Alvarez.] The set of vertices is the following:
\begin{verbatim}
(1 : -1 : -1 : -1 : 0 : 0)
(1 : 1 : -1 : -1 : 0 : 0)
(1 : -1 : 1 : -1 : 0 : 0)
(1 : 1 : 1 : -1 : 0 : 0)
(1 : 0 : 0 : 1 : 0 : 0)
(1 : 0 : 0 : 1 : 1 : 0)
(1 : 0 : 0 : 1 : 0 : 1)
\end{verbatim}
And this can be obtained as the join of a 2-cube with a 2-simplex. If we construct that in Polymake:
\begin{verbatim}
polytope > $p = join_polytopes(cube(2), simplex(2));

polytope > print($p->VERTICES);
1 -1 -1 -1 0 0
1 1 -1 -1 0 0
1 -1 1 -1 0 0
1 1 1 -1 0 0
1 0 0 1 0 0
1 0 0 1 1 0
1 0 0 1 0 1
\end{verbatim}

Thus, we can use the program to see the number of facets and the vertices in each facet:

\begin{verbatim}
polytope > print $p->N_FACETS;
polymake: used package cddlib
  Implementation of the double description method of Motzkin et al.
  Copyright by Komei Fukuda.
  http://www.ifor.math.ethz.ch/~fukuda/cdd_home/cdd.html

polymake: used package lrslib
  Implementation of the reverse search algorithm of Avis and Fukuda.
  Copyright by David Avis.
  http://cgm.cs.mcgill.ca/~avis/lrs.html

7
polytope > print $p->POINTS_IN_FACETS;
{0 2 4 5 6}
{0 1 4 5 6}
{1 3 4 5 6}
{2 3 4 5 6}
{0 1 2 3 5 6}
{0 1 2 3 4 6}
{0 1 2 3 4 5}
\end{verbatim}

If we focus now in the graph of the polytope, we can check the number of edges and we can also see it:
\begin{verbatim}
polytope > print $p->GRAPH->N_NODES;
7
polytope > print $p->GRAPH->N_EDGES;
19
polytope > print $p->GRAPH->VISUAL;
\end{verbatim}

\begin{figure}[!h]
\centering \includegraphics[width=\textwidth]{alex-alvarez-exercise-graph}
\caption{The graph of the polytope}
\end{figure}
Therefore, we can see that the last two vertices are not connected, but they are connected to all the other vertices.
\item[Exercise ?? (team members)] 
\end{description}



% Local Variables: 
% mode: latex
% TeX-master: "dag-upc"
% End: 

\chapter{Asymptotic f-vectors of families of polytopes}

\scribe{Cecilia Gir\'on}

In this section we are going to study the \textit{unimodality conjecture} which says that there exists an $l = P(L)\in\mathbb{N}$ such that $f_0\leq f_1 \leq \cdots\leq f_l \leq f_{l+1}\leq \cdots \leq f_{d-1}\leq f_d$. We woould like to know if it is true. 

First, we define the \textbf{$f$-vector} as the vector of the form $(f_0,f_{1},\cdots,f_{d-1})$ where $f_i$ is as defined before in (\ref{eq1}). We will say that it is a \textbf{flag $f$-vector} $(f_s)_s = [d]$ such that $f_s$ count the number of flags $F_{i_1} \subset F_{i_2} \subset \cdots \subset F_{i_k}$ where $s = \{i_1,i_2,\cdots, i_k \}$ and $\dim F_{i_k} = i_k$ \footnote{You can also read about $cd$-index}.

\bigskip
The unimodal conjecture described before is known to be false for simplical polytopes of dimension $d\geq 19$ and for non-simplicial polytopes of dimension $d \geq 8$. The following conjecture is not known to be false. 

\textbf{Restricted unimodal conjecture (Anders Bjorner)}: 
\begin{eqnarray*}
 f_0\leq f_1\leq \cdots \leq f_{\lfloor \frac{d-1}{4}}\rfloor\\
 f_{\lfloor \frac{3(d-1)}{4}\rfloor}\geq \cdots \geq f_{d-1}
\end{eqnarray*} 

Intuitively we are sure that there is no way this conjecture could be false, but there is not proof of this. We don't even know if $f_k\geq \frac{1}{10000}\min\{f_0,f_{d-1}\}$ is true.

\bigskip

\textbf{Exercises done during the lecture 11/11/2013. Each one includes one}
\textbf{Exercise 2b} (Borja and Cecilia). \textit{Using the simple form $n!\approx \left(\frac{n}{e}\right)^n$ of Stirling's formula, show that $\psi_d(x) : = d(1-x)+ \log  \binom {d} {xd}$ is asymptotically proportional to $1-x-x\log x - (1-x)\log(1-x)$, where  $\log = \log_2$ denotes the binary logarithm. Find an approximation to the maximum of this function on $(0,1)$}.

\bigskip
For the first part of the exercise, by applying the Stirling's formula, in the binomial for of the given function:
\[\binom{d}{xd} = \left(\frac{d}{xd^x (d(1-x))^{1-x}}\right)^d\]

Then, substituting in $\psi_d(x)$:

\begin{eqnarray*}
\psi_d(x) &=& d(1-x) + \log \left(\frac{d}{xd^x (d(1-x))^{1-x}}\right)^d \\
&=& d(1-x) + d\left(\log{d} - x\log{x} - x\log{d} - (1-x)\log{d} - (1-x) \log{(1-x)}\right)\\
&=& d (1-x-x\log x - (1-x)\log(1-x))
\end{eqnarray*}
Hence, $\psi_d(x)$ is asymptotically proportional to $1-x-x\log x - (1-x)\log(1-x)$
\bigskip

For the second part of the exercise, in order to find the maximum of the function
\begin{equation*}
f(x)=1-x-x\log(x)-(1-x)\log(1-x)
\end{equation*}
in (0,1) we will the points that have first derivative equal to 0, that is $x$ such that $f'(x)=0$, and computing $f'(x)$ we get:
\begin{equation*}
f'(x)=-1-(\log x+1)-(-\log(1-x)-1)=\log\left(\frac{1-x}{x}\right)-1
\end{equation*}
Now the points that make $f'(x)=0$ are the ones that make $\log(\frac{1-x}{x})=1$, which is the same as $x$ such that $\frac{1-x}{x}=e$, which translates into:
\begin{equation*}
x_{\max}=\frac{1}{e+1}
\end{equation*}
The shape of this function is a growing function from $x=0$ starting at $f(0)=1$ to $x=x_{\max}$, where it gets it's maximum, that is approximated by $f(x_max)\approx1.0414$  and then decreases to $0$ at $x=1$.


\section{Operations on polytopes}
\begin{itemize}
\item \textbf{Cartesian (direct) product} $P\times Q$.
\item \textbf{Direct sum} $P^d \oplus q^e \subset \mathbb{R}^{d+e}$.
\item $P*Q\subset \mathbb{R}^{d+e+1}$. It is like $\oplus$ but the subspaces are skew (i.e. affine and they have no point on common). For example $\square^1 *\square^1 = Pyr(P)$.
\end{itemize}

\bigskip
\noindent\textsc{Example}: Given $f_k(P)$, calculate the $k$-th entry of $Pyr(P)$:
\begin{eqnarray*}
f(P)&=&(f_0,f_1,\cdots, f_{d-1}\\
f_k(Pyr(P)) &=& (f_0 +1, f_1 + f_0, f_2+f_1, \cdots, f_{d-2}+ f_{d-3}, f_{d-1}+ f_{d-2}, 1+ f_{d-1})
\end{eqnarray*}
\begin{flushright}
$\clubsuit$
\end{flushright}

\bigskip
\begin{itemize}
\item  \textbf{Connected sum} $P^d\#Q^d$ where $P$ has as simplicial face $f$ and $Q$ has a simplicial face $G$. 
\end{itemize}

This last operation is used to join the asymptotic function $f(\square^d)$ and its dual $f(\diamondsuit ^d)$. To make it work, since $\square^d$ has no triangulations in its faces, it is enough to cut away one vertex and, this way, get a simplex. Merging both functions using the connected sum gives place to a new function which is a non-unimodal function.


% Local Variables: 
% mode: latex
% TeX-master: "dag-upc"
% End: 


\bibliographystyle{amsalpha}
\bibliography{dag}

\end{document}

%%% Local Variables: 
%%% mode: latex
%%% TeX-master: t
%%% End: 

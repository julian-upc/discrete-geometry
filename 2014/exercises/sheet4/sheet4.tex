\documentclass[11pt]{amsart}

\usepackage{a4wide}
\usepackage{paralist}
\usepackage{url}
\usepackage{nopageno}
\usepackage{bbm}
\usepackage[normalem]{ulem}
\usepackage{multicol}
\usepackage{color}

\newcommand{\cA}{\mathcal{A}}
\newcommand{\cS}{\mathcal{S}}
\newcommand{\R}{\mathbbm{R}}
\newcommand{\Z}{\mathbbm{Z}}
\newcommand{\VV}{\mathcal{V}}
\DeclareMathOperator{\conv}{conv}
\DeclareMathOperator{\cols}{cols}
\DeclareMathOperator{\Sl}{Sl}
\DeclareMathOperator{\igc}{igc}

\newcommand{\ojo}[1]{\textbf{\sffamily\boldmath{[#1]}}}
\newcommand{\defn}[1]{\textbf{\color{blue}#1}} %for highlighting defined terms in text
        
\newtheorem*{problemG}{Problem G}
\newtheorem*{problemG*}{Problem G$^\star$}

\begin{document}
\begin{center}
\textbf{\sffamily
   Discrete and Algorithmic Geometry }

\medskip
   Julian Pfeifle,
   UPC, 2014 \mbox{}
\end{center}

\bigskip

\begin{center}
  \textbf{\sffamily Sheet 4}

\bigskip
 due on Monday, December 22, 2014

\end{center}

\bigskip
\bigskip
\bigskip

\section*{Software}

The \defn{integral Gale complexity} of a polytope $P\subset\R^d$ with $n$~vertices is
\[
   \igc(P)
   \ = \
   \min\{\|G\|_\infty : G \subset\Z^e \text{ is a Gale diagram of } P\}, 
\]
where $e=n-d-1$, $\|\mathcal A\|_\infty = \max\{\|v\|_\infty : v\in\mathcal A\}$ and $\|v\|_\infty = \max\{|v_i|\}$ for $v=(v_1,\dots,v_e)$.

While the existence of nonrational polytopes shows that $\igc(P)=\infty$ is possible (since $\min\emptyset=\infty$), here we are concerned with the following problem:

\begin{problemG}
  For $e\in\Z_{\ge0}$ and $n, m\in\Z_{>0}$, determine 
  \[
     q(e,n,m)
     \ = \
     \#\left\{
     \begin{matrix}
       G\subset\Z^e : G \text{ is a Gale diagram of a polytope } \\
       \phantom{G\subset\Z^e}\!\!\text{with          $n$ vertices and } \igc(G)=m
     \end{matrix}
     \right\} \bigg/ \text{combinatorial equivalence}.
  \]
\end{problemG}

\noindent For example, $q(0,n,0) = 1$ and $q(1,n,m)=q(1,n,1)$ for all $m,n\ge1$. 

\bigskip
In more down-to-earth terms, we want to solve the following problem:

\begin{problemG*}
  Enumerate, up to combinatorial equivalence, all balanced   configurations~$\VV$ of $n$~vectors in~$\Z^e$ whose coordinates are   all at most~$m$ in absolute value, such that
\begin{enumerate}[\qquad\upshape(1)]
\item the maximum $m$ is achieved by some $v\in\VV$, 
\item and such that no hyperplane spanned by $e-1$~of the vectors   strictly separates exactly one vector from the others.
\end{enumerate}
\end{problemG*}

For this, recall 
\begin{itemize}
\item that a vector configuration $\VV = (v_1,\dots,v_n)$ is \defn{balanced} if $\sum_i v_i=0$;
\item that no hyperplane defined by $e-1$~elements of~$\VV$ separates   exactly one vector from the others iff the Gale dual of~$\VV$ is in   convex position;
\item and that two vector configurations are \defn{combinatorially equivalent} if they define the same oriented matroid.
\end{itemize}

\bigskip
Your job is to write a function in the \texttt{polymake} framework that calculates $q(e,n,m)$. Some considerations to keep in mind:
\begin{itemize}
\item Correctness is more important than efficiency, but efficiency is supremely important.
\item 
\end{itemize}

\end{document}
